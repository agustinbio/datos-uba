\documentclass[10pt,a4paper]{article}

\usepackage[spanish,activeacute,es-tabla]{babel}
\usepackage[utf8]{inputenc}
\usepackage{ifthen}
\usepackage{listings}
\usepackage{dsfont}
\usepackage{subcaption}
\usepackage{amsmath}
\usepackage[strict]{changepage}
\usepackage[top=1cm,bottom=2cm,left=1cm,right=1cm]{geometry}%
\usepackage{color}%
\newcommand{\tocarEspacios}{%
	\addtolength{\leftskip}{3em}%
	\setlength{\parindent}{0em}%
}

% Especificacion de procs

\newcommand{\In}{\textsf{in }}
\newcommand{\Out}{\textsf{out }}
\newcommand{\Inout}{\textsf{inout }}

\newcommand{\encabezadoDeProc}[4]{%
	% Ponemos la palabrita problema en tt
	%  \noindent%
	{\normalfont\bfseries\ttfamily proc}%
	% Ponemos el nombre del problema
	\ %
	{\normalfont\ttfamily #2}%
	\
	% Ponemos los parametros
	(#3)%
	\ifthenelse{\equal{#4}{}}{}{%
		% Por ultimo, va el tipo del resultado
		\ : #4}
}

\newenvironment{proc}[4][res]{%
	
	% El parametro 1 (opcional) es el nombre del resultado
	% El parametro 2 es el nombre del problema
	% El parametro 3 son los parametros
	% El parametro 4 es el tipo del resultado
	% Preambulo del ambiente problema
	% Tenemos que definir los comandos requiere, asegura, modifica y aux
	\newcommand{\requiere}[2][]{%
		{\normalfont\bfseries\ttfamily requiere}%
		\ifthenelse{\equal{##1}{}}{}{\ {\normalfont\ttfamily ##1} :}\ %
		\{\ensuremath{##2}\}%
		{\normalfont\bfseries\,\par}%
	}
	\newcommand{\asegura}[2][]{%
		{\normalfont\bfseries\ttfamily asegura}%
		\ifthenelse{\equal{##1}{}}{}{\ {\normalfont\ttfamily ##1} :}\
		\{\ensuremath{##2}\}%
		{\normalfont\bfseries\,\par}%
	}
	\renewcommand{\aux}[4]{%
		{\normalfont\bfseries\ttfamily aux\ }%
		{\normalfont\ttfamily ##1}%
		\ifthenelse{\equal{##2}{}}{}{\ (##2)}\ : ##3\, = \ensuremath{##4}%
		{\normalfont\bfseries\,;\par}%
	}
	\renewcommand{\pred}[3]{%
		{\normalfont\bfseries\ttfamily pred }%
		{\normalfont\ttfamily ##1}%
		\ifthenelse{\equal{##2}{}}{}{\ (##2) }%
		\{%
		\begin{adjustwidth}{+5em}{}
			\ensuremath{##3}
		\end{adjustwidth}
		\}%
		{\normalfont\bfseries\,\par}%
	}
	
	\newcommand{\res}{#1}
	\vspace{1ex}
	\noindent
	\encabezadoDeProc{#1}{#2}{#3}{#4}
	% Abrimos la llave
	\par%
	\tocarEspacios
}
{
	% Cerramos la llave
	\vspace{1ex}
}

\newcommand{\aux}[4]{%
	{\normalfont\bfseries\ttfamily\noindent aux\ }%
	{\normalfont\ttfamily #1}%
	\ifthenelse{\equal{#2}{}}{}{\ (#2)}\ : #3\, = \ensuremath{#4}%
	{\normalfont\bfseries\,;\par}%
}

\newcommand{\pred}[3]{%
	{\normalfont\bfseries\ttfamily\noindent pred }%
	{\normalfont\ttfamily #1}%
	\ifthenelse{\equal{#2}{}}{}{\ (#2) }%
	\{%
	\begin{adjustwidth}{+2em}{}
		\ensuremath{#3}
	\end{adjustwidth}
	\}%
	{\normalfont\bfseries\,\par}%
}

% Tipos

\newcommand{\nat}{\ensuremath{\mathds{N}}}
\newcommand{\ent}{\ensuremath{\mathds{Z}}}
\newcommand{\float}{\ensuremath{\mathds{R}}}
\newcommand{\bool}{\ensuremath{\mathsf{Bool}}}
\newcommand{\cha}{\ensuremath{\mathsf{Char}}}
\newcommand{\str}{\ensuremath{\mathsf{String}}}

% Logica

\newcommand{\True}{\ensuremath{\mathrm{true}}}
\newcommand{\False}{\ensuremath{\mathrm{false}}}
\newcommand{\Then}{\ensuremath{\rightarrow}}
\newcommand{\Iff}{\ensuremath{\leftrightarrow}}
\newcommand{\implica}{\ensuremath{\longrightarrow}}
\newcommand{\IfThenElse}[3]{\ensuremath{\mathsf{if}\ #1\ \mathsf{then}\ #2\ \mathsf{else}\ #3\ \mathsf{fi}}}
\newcommand{\yLuego}{\land _L}
\newcommand{\oLuego}{\lor _L}
\newcommand{\implicaLuego}{\implica _L}

\newcommand{\cuantificador}[5]{%
	\ensuremath{(#2 #3: #4)\ (%
		\ifthenelse{\equal{#1}{unalinea}}{
			#5
		}{
			$ % exiting math mode
			\begin{adjustwidth}{+2em}{}
				$#5$%
			\end{adjustwidth}%
			$ % entering math mode
		}
		)}
}

\newcommand{\existe}[4][]{%
	\cuantificador{#1}{\exists}{#2}{#3}{#4}
}
\newcommand{\paraTodo}[4][]{%
	\cuantificador{#1}{\forall}{#2}{#3}{#4}
}

%listas

\newcommand{\TLista}[1]{\ensuremath{seq \langle #1\rangle}}
\newcommand{\lvacia}{\ensuremath{[\ ]}}
\newcommand{\lv}{\ensuremath{[\ ]}}
\newcommand{\longitud}[1]{\ensuremath{|#1|}}
\newcommand{\cons}[1]{\ensuremath{\mathsf{addFirst}}(#1)}
\newcommand{\indice}[1]{\ensuremath{\mathsf{indice}}(#1)}
\newcommand{\conc}[1]{\ensuremath{\mathsf{concat}}(#1)}
\newcommand{\cab}[1]{\ensuremath{\mathsf{head}}(#1)}
\newcommand{\cola}[1]{\ensuremath{\mathsf{tail}}(#1)}
\newcommand{\sub}[1]{\ensuremath{\mathsf{subseq}}(#1)}
\newcommand{\en}[1]{\ensuremath{\mathsf{en}}(#1)}
\newcommand{\cuenta}[2]{\mathsf{cuenta}\ensuremath{(#1, #2)}}
\newcommand{\suma}[1]{\mathsf{suma}(#1)}
\newcommand{\twodots}{\ensuremath{\mathrm{..}}}
\newcommand{\masmas}{\ensuremath{++}}
\newcommand{\matriz}[1]{\TLista{\TLista{#1}}}
\newcommand{\seqchar}{\TLista{\cha}}

\renewcommand{\lstlistingname}{Código}
\lstset{% general command to set parameter(s)
	language=Java,
	morekeywords={endif, endwhile, skip},
	basewidth={0.47em,0.40em},
	columns=fixed, fontadjust, resetmargins, xrightmargin=5pt, xleftmargin=15pt,
	flexiblecolumns=false, tabsize=4, breaklines, breakatwhitespace=false, extendedchars=true,
	numbers=left, numberstyle=\tiny, stepnumber=1, numbersep=9pt,
	frame=l, framesep=3pt,
	captionpos=b,
}

\usepackage{caratula,amssymb} % Version modificada para usar las macros de algo1 de ~> https://github.com/bcardiff/dc-tex
\newcommand{\llamarAux}[1]{\text{\normalfont\ttfamily #1}}
\newcommand{\llamarPred}[1]{\text{\normalfont\ttfamily #1}}
\newcommand{\refVector}[1]{\text{#1}}
\newcommand{\refVariable}[1]{\text{#1}}
\newcommand{\recursos}{\refVector{recursos}}
\newcommand{\cooperan}{\refVector{cooperan}}
\newcommand{\cooperancero}{\refVector{Cooperan\textsubscript{0}}}
\newcommand{\resultado}{\refVector{res}}
\newcommand{\trayectoria}{\llamarAux{trayectoria}}
\newcommand{\trayectoriauno}{\refVector{trayectoria\textsubscript{orig}}}
\newcommand{\trayectoriados}{\refVector{trayectoria\textsubscript{neg}}}
\newcommand{\trayectoriamax}{\refVector{trayectoria\textsubscript{max}}}
\newcommand{\trayectoriaposible}{\refVector{trayectoria\textsubscript{posible}}}
\newcommand{\trayectorias}{\refVector{trayectorias}}
\newcommand{\trayectoriascero}{\refVector{Trayectorias\textsubscript{0}}}
\newcommand{\apuestas}{\refVector{apuestas}}
\newcommand{\pagos}{\refVector{pagos}}
\newcommand{\eventos}{\refVector{eventos}}
\newcommand{\individuo}{\refVector{individuo}}
%\newcommand{\ind}{\ensuremath{i}}
\newcommand{\ind}{\refVector{ind}}
\newcommand{\indcoop}{\refVector{indcoop}}
\newcommand{\apuestascero}{\refVector{Apuestas\textsubscript{0}}}
\newcommand{\apuestaposible}{\refVector{apuesta\textsubscript{posible}}}

\titulo{Trabajo Práctico N\textsuperscript{\underline{o}} 1}
\subtitulo{Especificación y WP}

\fecha{\today}

\materia{Algoritmos y Estructuras de Datos}
\grupo{Integrantes:}

\integrante{Bianchi, Bruno}{699/22}{bruno@brunobianchi.com.ar}
\integrante{Castro Vivoda, Iván}{43/22}{icastrovivoda11@gmail.com}
\integrante{Chen, Santiago}{114/23}{chensantiago000@gmail.com}
\integrante{Herrera, Agustín}{609/10}{agustin@inventati.org}
% Pongan cuantos integrantes quieran

% Declaramos donde van a estar las figuras
% No es obligatorio, pero suele ser comodo
\graphicspath{{../static/}}

\begin{document}

\maketitle

\section{Especificación}
\subsection{redistribucionDeLosFrutos}

\begin{proc}{redistribucionDeLosFrutos}{\In recursos : \TLista{\float}, \In cooperan : \TLista{\bool}}{\TLista{\float}}
	%    \modifica{parametro1, parametro2,..}
	\requiere{|\recursos|=|\cooperan|}
	\requiere{|\recursos|>0}
	\requiere{\paraTodo[unalinea]{i}{\ent}{0\leq i <|\recursos| \implicaLuego \recursos [i] > 0}}
	\asegura{|\resultado|=|\recursos|}
	\asegura{\paraTodo[unalinea]{i}{\ent}{0\leq i <|\resultado| \implicaLuego \resultado [i] = \recursos [i]*(1-\llamarAux{coopera}(\cooperan[i])) + \llamarAux{fracRecurso$($\recursos,\cooperan$)$}}}
\end{proc}
	\aux{fracRecurso}{recursos: \TLista{\float}, cooperan \TLista{\bool}}{\float}{\frac{\sum_{i = 0}^{|\recursos|-1} \recursos[i]*\llamarAux{coopera}(\cooperan[i])}{|\recursos|}}
	\aux{coopera}{coop: \bool}{\nat}{\IfThenElse {\refVariable{coop = \True}}{1}{0}}
	
\subsection{trayectoriaDeLosFrutosIndividualesALargoPlazo}	
\begin{proc}{trayectoriaDeLosFrutosIndividualesALargoPlazo}{\Inout trayectorias: \TLista{\TLista{\float}}, \In cooperan: \TLista{\bool},
 \In apuestas: \TLista{\TLista{\float}}, \In pagos: \TLista{\TLista{\float}}, \In eventos: \TLista{\TLista{\nat}} }{}	
	\requiere{\trayectorias = \trayectoriascero}
	\requiere{\paraTodo[unalinea]{i}{\ent}{0\leq i <|\trayectoriascero| \implicaLuego |\trayectoriascero [i]| = 1}} %la trayectoria inicial tiene un unico tiempo
	\requiere{|\trayectoriascero| = |\cooperan| = |\apuestas| = |\pagos| = |\eventos|} %misma cantidad de individuos en todas las secuencias
	\requiere{|\trayectoriascero| > 0}	%al menos un individuo
	\requiere{\paraTodo[unalinea]{i}{\ent}{0\leq i <|\eventos| \implicaLuego |\eventos[i]| > 0}} %tiene que existir al menos 1 tiempo (por cada individuo)
	\requiere{\paraTodo[unalinea]{i}{\ent}{0\leq i <|\trayectoriascero| \implicaLuego \trayectoriascero [i][0] > 0}} %los recursos iniciales son positivos
	\requiere{\paraTodo[unalinea]{i}{\ent}{0\leq i <|\apuestas| \implicaLuego \paraTodo[unalinea]{j}{\ent}{0\leq j<|\apuestas [i]| \implicaLuego \apuestas [i][j] \geq 0}}} %las apuestas son números positivos o cero
	\requiere{\paraTodo[unalinea]{i}{\ent}{0\leq i <|\apuestas| \implicaLuego \sum_{j=0}^{|\apuestas[i]|-1} \apuestas [i][j] = 1}} %las apuestas son proporciones que suman 1 (por individuo)
	\requiere{\paraTodo[unalinea]{i}{\ent}{0\leq i <|\apuestas| \implicaLuego |\apuestas [i]| = |\pagos [i]| > 1}} %el largo de apuestas y de pagos es la cantidad de eventos y como mínimo tiene que haber 2 eventos
	\requiere{\paraTodo[unalinea]{i}{\ent}{0\leq i <|\pagos| \implicaLuego \paraTodo[unalinea]{j}{\ent}{0\leq j<|\pagos [i]| \implicaLuego \pagos[i][j] > 0}}} %todos los pagos son positivos

	\requiere{\paraTodo[unalinea]{i}{\ent}{0\leq i <|\eventos| \implicaLuego \paraTodo[unalinea]{j}{\ent}{0\leq j<|\eventos [i]| \implicaLuego 0 \leq \eventos [i][j] < |\apuestas [i]|}}} %el evento que le ha tocado a un individuo tiene que ser un evento posible (no debe exceder el largo de la secuencia apuestas o pagos
    \asegura{\llamarPred{esTrayectoria}(\trayectorias, \trayectoriascero, \cooperan, \apuestas,\pagos,\eventos)}
\end{proc}	
\pred{esTrayectoria}{trayectorias: \TLista{\TLista{\float}}, \trayectoriascero: \TLista{\TLista{\float}}, cooperan: \TLista{\bool}, apuestas: \TLista{\TLista{\float}}, pagos: \TLista{\TLista{\float}}, eventos: \TLista{\TLista{\nat}}}{
    |\trayectorias|=|\trayectoriascero| \yLuego
    \\\paraTodo[unalinea]{i}{\ent}{0\leq i < |\trayectorias| \implicaLuego |\trayectorias[i]| = |\eventos[i]|+1} \yLuego
    \\\paraTodo[unalinea]{i}{\ent}{0\leq i < |\trayectorias| \implicaLuego \trayectorias[i][0] = \trayectoriascero[i][0]} \yLuego
 \\\paraTodo[unalinea]{i,j}{\ent}{0\leq i < |\trayectorias| \land 0 \leq j < |\eventos[i]| \implicaLuego \\\trayectorias[i][j+1] = \trayectorias[i][j] * \apuestas[i][\eventos[i][j]] * \pagos[i][\eventos[i][j]] * (1-\llamarAux{coopera}(\cooperan[i])) + \frac{\sum_{k = 0}^{|\trayectorias|-1} {\trayectorias[k][j] * \apuestas[k][\eventos[k][j]] * \pagos[k][\eventos[k][j]]}*\llamarAux{coopera}(\cooperan[k])}{|\trayectorias|}}}	
	
%\aux{frutosIndividuales}{\In \trayectorias: \TLista{\TLista{\float}, \In \apuestas: \TLista{\TLista{\float}}, \In \pagos: \TLista{\TLista{\float}}, \In \eventos: \TLista{\TLista{\nat}}}} {\float}{\trayectorias[i][j] * \apuestas[i][\eventos[i][j]] * \pagos[i][\eventos[i][j]]}

\aux{coopera}{coop: \bool}{\nat}{\IfThenElse {\refVariable{coop} = \True}{1}{0}}	

\subsection{trayectoriaExtrañaEscalera}	
\begin{proc}{trayectoriaExtrañaEscalera}{\In \refVector{trayectoria}: \TLista{\float}}{\bool}
	\requiere{|\refVector{trayectoria}| > 1}
	\asegura{res = \True \Iff \llamarAux{cantidadMaximoLocal}(\refVector{trayectoria} )= 1}	
\end{proc}
\aux{cantidadMaximoLocal}{\In s: \TLista{\float}}{\ent}{\\\IfThenElse{s[0]>s[1]}{1}{0}+\sum_{i=1}^{|s|-2}(\IfThenElse{s[i]>s[i-1]\land s[i]>s[i+1]}{1}{0})+\IfThenElse{s[|s|-1]>s[|s|-2]}{1}{0}}


\subsection{individuoDecideSiCooperarONo}	
\begin{proc}{individuoDecideSiCooperarONo}{\In individuo: \nat, \In recursos: \TLista{\float}, \Inout cooperan: \TLista{\bool}, \In apuestas: \TLista{\TLista{\float}},\\\In pagos \TLista{\TLista{\float}}, \In eventos: \TLista{\TLista{\nat}} }{}
	\requiere{\cooperan = \cooperancero}
	\requiere{|\recursos| = |\cooperancero| = |\apuestas| = |\pagos| = |\eventos|}  %misma cantidad de individuos en todas las secuencias
	\requiere{|\cooperancero| > 1} %tiene que haber al menos 2 individuos (para que tenga sentido el procedimiento)
	\requiere{0\leq\individuo<|\cooperancero|} %individuo está en rango
	\requiere{\paraTodo[unalinea]{i}{\ent}{0\leq i <|\eventos| \implicaLuego |\eventos[i]| > 0}} %tiene que existir al menos 1 tiempo (por cada individuo)
	\requiere{\paraTodo[unalinea]{i}{\ent}{0\leq i <|\recursos| \implicaLuego \recursos [i] > 0}} %los recursos iniciales son positivos
	\requiere{\paraTodo[unalinea]{i}{\ent}{0\leq i <|\apuestas| \implicaLuego \paraTodo[unalinea]{j}{\ent}{0\leq j<|\apuestas [i]| \implicaLuego \apuestas [i][j] \geq 0}}} %las apuestas son números positivos o cero
	\requiere{\paraTodo[unalinea]{i}{\ent}{0\leq i <|\apuestas| \implicaLuego \sum_{j=0}^{|\apuestas[i]|-1} \apuestas [i][j] = 1}} %las apuestas son proporciones que suman 1 (por individuo)
	\requiere{\paraTodo[unalinea]{i}{\ent}{0\leq i <|\apuestas| \implicaLuego |\apuestas [i]| = |\pagos [i]| > 1}} %el largo de apuestas y de pagos es la cantidad de ev\entos y como mínimo tiene que haber 2 eventos
	\requiere{\paraTodo[unalinea]{i}{\ent}{0\leq i <|\pagos| \implicaLuego \paraTodo[unalinea]{j}{\ent}{0\leq j<|\pagos [i]| \implicaLuego \pagos[i][j] > 0}}} %todos los pagos son positivos

	\requiere{\paraTodo[unalinea]{i}{\ent}{0\leq i <|\eventos| \implicaLuego \paraTodo[unalinea]{j}{\ent}{0\leq j<|\eventos [i]| \implicaLuego 0 \leq \eventos [i][j] < |\apuestas [i]|}}} %el evento que le ha tocado a un individuo tiene que ser un evento posible (no debe exceder el largo de la secuencia apuestas o pagos

	\asegura{|\cooperan| = |\cooperancero|}
	\asegura{\paraTodo[unalinea]{i}{\ent}{0\leq i <|\cooperan| \land i \neq \individuo \implicaLuego \cooperan[i] = \cooperancero[i]} \yLuego
	\\\existe[unalinea]{\trayectoriauno}{\TLista{\TLista{\float}}}{\\\llamarPred{esTrayectoria}(\trayectoriauno, \individuo, \cooperancero[\individuo],\recursos, \cooperancero, \apuestas,\pagos,\eventos)}\land \\\existe[unalinea]{\trayectoriados}{\TLista{\TLista{\float}}}{\\\llamarPred{esTrayectoria}(\trayectoriados, \individuo, \neg\cooperancero[\individuo],\recursos, \cooperancero, \apuestas,\pagos,\eventos)} \yLuego\\ \trayectoriauno[\individuo][|\eventos|]\geq\trayectoriados[\individuo][|\eventos|] \Iff \cooperan[\individuo] = \cooperancero[\individuo]}
	%(\existe[unalinea]{\trayectoriauno}{\TLista{\TLista{\float}}}{\\\llamarPred{esTrayectoria}(\trayectoriauno, \individuo, \cooperancero[\individuo],\recursos, \cooperancero, \apuestas,\pagos,\eventos)}\land \\\existe[unalinea]{\trayectoriados}{\TLista{\TLista{\float}}}{\\\llamarPred{esTrayectoria}(\trayectoriados, \individuo, \neg\cooperancero[\individuo],\recursos, \cooperancero, \apuestas,\pagos,\eventos)} \yLuego\\ (\trayectoriauno[\individuo][|\eventos|]<\trayectoriados[\individuo][|\eventos|]) \implica\\ \\\land \cooperan[\individuo] = \neg \cooperancero[\individuo])}
	\end{proc}
\pred{esTrayectoria}{\trayectorias: \TLista{\TLista{\float}}, \ind: \nat, \indcoop: \bool , recursos: \TLista{\float}, cooperan: \TLista{\bool}, apuestas: \TLista{\TLista{\float}}, pagos \TLista{\TLista{\float}}, eventos: \TLista{\TLista{\nat}}}
{
	\paraTodo[unalinea]{i}{\ent}{0\leq i < |\trayectorias| \implicaLuego \trayectorias[i][0] = \recursos[i] \land \paraTodo[unalinea]{j}{\ent}{0 \leq j < |\eventos[i]| \implicaLuego \\\trayectorias[i][j+1] = \trayectorias[i][j] * \apuestas[i][\eventos[i][j]] * \pagos[i][\eventos[i][j]] * \\(1-\llamarAux{cooperaInd}(i,\ind,\indcoop,\cooperan)) + \frac{1}{|\cooperan|}*\\\sum_{k = 0}^{|\cooperan|-1} {\trayectorias[k][j] * \apuestas[k][\eventos[k][j]] * \pagos[k][\eventos[k][j]]}*\llamarAux{cooperaInd}(k,\ind,\indcoop,\cooperan)}}
}
\aux{coopera}{coop: \bool}{\nat}{\IfThenElse {\refVariable{coop} = \True}{1}{0}}
\aux{cooperaInd}{k: \nat, \ind: \nat, \indcoop: \bool, \cooperan: \TLista{\bool}}{\nat}
{\\\IfThenElse {k = \ind}{\llamarAux{coopera}(\indcoop)}{\llamarAux{coopera}(\cooperan[k])}}

\subsection{individuoActualizaApuesta}	
\begin{proc}{individuoActualizaApuesta}{\In individuo: \nat, \In recursos: \TLista{\float}, \In cooperan: \TLista{\bool}, \Inout apuestas: \TLista{\TLista{\float}},\\\In pagos \TLista{\TLista{\float}}, \In eventos: \TLista{\TLista{\nat}} }{}
	\requiere{\apuestas = \apuestascero}
	\requiere{|\recursos| = |\cooperan| = |\apuestascero| = |\pagos| = |\eventos|}  %misma cantidad de individuos en todas las secuencias
	\requiere{|\apuestascero| > 0} %tiene que haber al menos 1 individuo
	\requiere{0\leq\individuo<|\apuestascero|} %individuo está en rango
	\requiere{\paraTodo[unalinea]{i}{\ent}{0\leq i <|\eventos| \implicaLuego |\eventos[i]| > 0}} %tiene que existir al menos 1 tiempo (por cada individuo)
	\requiere{\paraTodo[unalinea]{i}{\ent}{0\leq i <|\recursos| \implicaLuego \recursos [i] > 0}} %los recursos iniciales son positivos
	\requiere{\paraTodo[unalinea]{i}{\ent}{0\leq i <|\apuestascero| \implicaLuego \paraTodo[unalinea]{j}{\ent}{0\leq j<|\apuestascero [i]| \implicaLuego \apuestascero [i][j] \geq 0}}} %las apuestas son números positivos o cero
	\requiere{\paraTodo[unalinea]{i}{\ent}{0\leq i <|\apuestascero| \implicaLuego \sum_{j=0}^{|\apuestascero[i]|-1} \apuestascero [i][j] = 1}} %las apuestas son proporciones que suman 1 (por individuo)
	\requiere{\paraTodo[unalinea]{i}{\ent}{0\leq i <|\apuestascero| \implicaLuego |\apuestascero [i]| = |\pagos [i]| > 1}} %el largo de apuestas y de pagos es la cantidad de eventos y como mínimo tiene que haber 2 eventos
	\requiere{\paraTodo[unalinea]{i}{\ent}{0\leq i <|\pagos| \implicaLuego \paraTodo[unalinea]{j}{\ent}{0\leq j<|\pagos [i]| \implicaLuego \pagos[i][j] > 0}}} %todos los pagos son positivos

	\requiere{\paraTodo[unalinea]{i}{\ent}{0\leq i <|\eventos| \implicaLuego \paraTodo[unalinea]{j}{\ent}{0\leq j<|\eventos [i]| \implicaLuego 0 \leq \eventos [i][j] < |\apuestascero [i]|}}} %el evento que le ha tocado a un individuo tiene que ser un evento posible (no debe exceder el largo de la secuencia apuestas o pagos

	\asegura{|\apuestas| = |\apuestascero|}
	\asegura{\existe[unalinea]{\trayectoriamax}{\TLista{\TLista{\float}}}{\llamarPred{esTrayectoria}(\trayectoriamax, \recursos,\cooperan,\apuestas,\pagos,\eventos)} \yLuego
	\paraTodo[unalinea]{\apuestaposible}{\TLista{\TLista{\float}}}{\llamarPred{esApuestaPosible}(\apuestaposible, \apuestascero, \individuo) \implicaLuego
	\\\existe[unalinea]{\trayectoriaposible}{\TLista{\TLista{\float}}}{\llamarPred{esTrayectoria}(\trayectoriaposible, \recursos,\cooperan,\apuestaposible,\pagos,\eventos)
	\\\yLuego \trayectoriamax[\individuo][|\eventos|] \geq \trayectoriaposible[\individuo][|\eventos|]}}}
	\end{proc}
%
\pred{esApuestaPosible}{\apuestaposible: \TLista{\TLista{\float}}, \apuestascero: \TLista{\TLista{\float}}, \individuo: \nat}{
(|\apuestaposible| = |\apuestascero| \yLuego \\\paraTodo[unalinea]{i}{\ent}{0\leq i<|\apuestaposible| \land i \neq \individuo \implicaLuego \apuestaposible [i] = \apuestascero[i]} \land \\\sum_{k=0}^{|\apuestaposible[\individuo]|-1}\apuestaposible[\individuo][k]=1 \land \\\paraTodo[unalinea]{j}{\ent}{0\leq j<|\apuestaposible[\individuo]| \implicaLuego \apuestaposible[\individuo][j]\geq 0})}

\pred{esTrayectoria}{\trayectorias: \TLista{\TLista{\float}}, recursos: \TLista{\float}, cooperan: \TLista{\bool},\\ apuestas: \TLista{\TLista{\float}}, pagos \TLista{\TLista{\float}}, eventos: \TLista{\TLista{\nat}}}
	{
		\paraTodo[unalinea]{i}{\ent}{0\leq i < |\trayectorias| \implicaLuego \trayectorias[i][0] = \recursos[i] \land \paraTodo[unalinea]{j}{\ent}{0 \leq j < |\eventos[i]| \implicaLuego \\\trayectorias[i][j+1] = \trayectorias[i][j] * \apuestas[i][\eventos[i][j]] * \pagos[i][\eventos[i][j]] * \\(1-\llamarAux{coopera}(\cooperan[i])) + \frac{1}{|\cooperan|}*\\\sum_{k = 0}^{|\cooperan|-1} {\trayectorias[k][j] * \apuestas[k][\eventos[k][j]] * \pagos[k][\eventos[k][j]]}*\llamarAux{coopera}(\cooperan[k])}}
	}
\aux{coopera}{coop: \bool}{\nat}{\IfThenElse {\refVariable{coop} = \True}{1}{0}}

\newpage

\section{Demostraciones de correctitud}

Verificamos formalmente la correctitud del programa y en un primer paso definimos el invariante de ciclo como:

\begin{equation}
    \resizebox{.9\hsize}{!}{${\it I} = 0 \leq i \leq |eventos| \yLuego res = recurso(apuesta_c * pago_c)^{\#apariciones(subseq(eventos,0,i),True)} *(apuesta_s * pago_s)^{\#apariciones(subseq(eventos,0,i),False)}$}
\end{equation}

Donde apariciones: \\

\aux{apariciones}{\In s: \TLista{T}}{\ent}{ \sum_{i = 0}^{|s|-1} (\IfThenElse {s[i] = e}{1}{0})}

\vspace{0.3cm}
Definimos luego la precondición del ciclo
\begin{equation}
    P_c:\{ recurso > 0 \land res=recurso \land i=0 \}
\end{equation}

\subsection{Axioma 1}
Axioma 1: Verificamos que $P_c \implies {\it I}$
\vspace{0.3cm}

\begin{equation}
    \begin{split}
        & recurso > 0 \land res=recurso \land i=0  \implies \\
        & \resizebox{.9\hsize}{!}{$0 \leq 0 \leq |eventos| \yLuego res = recurso(apuesta_c * pago_c)^{\#apariciones(subseq(eventos,0,0),True)}*(apuesta_s * pago_s)^{\#apariciones(subseq(eventos,0,0),False)}$}
    \end{split}
\end{equation}

\begin{equation}
    recurso > 0 \land res=recurso \land i=0 \implies 0 \leq 0 \leq |eventos| \yLuego res = recurso(apuesta_c * pago_c)^0*(apuesta_s * pago_s)^0
\end{equation}

Lo cual es trivialmente cierto, $\therefore P_c \implies {\it I}$
\vspace{0.3cm}

\subsection{Axioma 2}
Axioma 2: Verificamos que $\{\it I \land B \}$ S $\{\it I \}$ \\

\begin{equation}
    \begin{gathered}
        wp(S,{\it I}) \equiv wp(if...endif; i:=i+1, I) \\
        wp(S,{\it I}) \equiv wp(if...endif, wp(i:=i+1, I)) \\
        wp(S,{\it I}) \equiv wp(if...endif, def(i+1) \land {\it I}_{i+1}^i)\\
        %A partir de acá no estoy seguro de que sea correcto, hay que añadir la definicion de i+1, que está entre 0 < i < |eventos|
        \resizebox{.9\hsize}{!}{$wp(S,{\it I})\equiv (-1 \leq i \leq |eventos| -1) \yLuego ((eventos[i]=True \land wp(res=res*apuesta_c*pago_c,{\it I}_{i+1}^i)) \lor (eventos[i]=False \land wp(res=res*apuesta_s*pago_s,{\it I}_{i+1}^i)))$}
    \end{gathered}
\end{equation}

\begin{equation}
    \begin{gathered}
        wp(S,{\it I}) \equiv (-1 \leq i \leq |eventos|-1) \yLuego \\
        \resizebox{.9\hsize}{!}{$((eventos[i]=True \land res*(apuesta_c*pago_c)=recurso*(apuesta_c*pago_c)^{\#apariciones(subseq (eventos,0,i+1),True)}*(apuesta_s * pago_s)^{\#apariciones(subseq(eventos,0,i+1),False)})$} \\
        \resizebox{.9\hsize}{!}{$\lor (eventos[i]=False \land res*(apuesta_s*pago_s)=recurso*(apuesta_c*pago_c)^{\#apariciones(subseq (eventos,0,i+1),True)}*(apuesta_s * pago_s)^{\#apariciones(subseq(eventos,0,i+1),False)}))$}\\
    \end{gathered}
\end{equation}

\begin{equation}
    \begin{gathered}
        \resizebox{.9\hsize}{!}{$wp(S,{\it I}) \equiv (-1 \leq i \leq |eventos|-1) \yLuego ((eventos[i]=True \yLuego True) \lor (eventos[i]=False \yLuego True))$}
    \end{gathered}
\end{equation}

Verificamos que ${\it I} \land B \implies wp(S,{\it I})$

\begin{equation}
    \begin{gathered}
        \resizebox{.9\hsize}{!}{$0 \leq i < |eventos| \yLuego res = recurso(apuesta_c * pago_c)^{\#apariciones(subseq(eventos,0,i),True)}*(apuesta_s * pago_s)^{\#apariciones(subseq(eventos,0,i),False)} \implies $}\\
        \resizebox{.9\hsize}{!}{$(-1 \leq i \leq |eventos|-1) \yLuego ((eventos[i]=True \yLuego True) \lor (eventos[i]=False \yLuego True))$} \\
        0 \leq i < |eventos| \implies (-1 \leq i \leq |eventos|-1) \\
        0 \leq i < |eventos| \implies (-1 \leq i < |eventos|)
    \end{gathered}
\end{equation}

$\therefore \{\it I \land B \}$ S $\{\it I \}$

\vspace{0.3cm}

\subsection{Axioma 3}
Axioma 3: Verificamos que ${\it I} \land \neg B \implies Q_c$

\begin{equation}
    B: \{ i < |eventos|\}
\end{equation}

\begin{equation}
    Q_c: \{ res = recurso(apuesta_c * pago_c)^{\#apariciones(eventos,True)} *(apuesta_s * pago_s)^{\#apariciones(eventos,False)} \}
\end{equation}


\begin{equation}
    \begin{aligned}
        &\resizebox{.9\hsize}{!}{$(0 \leq i \leq |eventos| \yLuego res = recurso(apuesta_c * pago_c)^{\#apariciones(subseq(eventos,0,i),True)} *(apuesta_s * pago_s)^{\#apariciones(subseq(eventos,0,i),False)})$}\\
        & \land \neg (i < |eventos|) \implies res = recurso(apuesta_c * pago_c)^{\#apariciones(eventos,True)} *(apuesta_s * pago_s)^{\#apariciones(eventos,False)}
    \end{aligned}
\end{equation}

$\neg (i < |eventos|) = i \geq |eventos|$ pero $i \leq |eventos| \therefore i = |eventos|$

\vspace{0.3cm}

Luego:

\begin{equation}
    \begin{aligned}
    &\resizebox{.9\hsize}{!}{$res = recurso(apuesta_c * pago_c)^{\#apariciones(subseq(eventos,0,|eventos|),True)} *(apuesta_s * pago_s)^{\#apariciones(subseq(eventos,0,|eventos|),False)}$} \\
    &\implies res = recurso(apuesta_c * pago_c)^{\#apariciones(eventos,True)} *(apuesta_s * pago_s)^{\#apariciones(eventos,False)}
    \end{aligned}
\end{equation}

Pero $subseq(eventos,0,|eventos|) = eventos$

\begin{equation}
    \begin{aligned}
    &res = recurso(apuesta_c * pago_c)^{\#apariciones(eventos,True)} *(apuesta_s * pago_s)^{\#apariciones(eventos,False)} \\
    &\implies res = recurso(apuesta_c * pago_c)^{\#apariciones(eventos,True)} *(apuesta_s * pago_s)^{\#apariciones(eventos,False)} \\
    \end{aligned}
\end{equation}
Que es a lo que se quería llegar. Por lo tanto, ${\it I} \land \neg B \implies Q_c$ \\

Queda así demostrado que $P_c \implies wp(S,Q_c)$

\vspace{0.3cm}

\subsection{Axioma 4}
Axioma 4: Verificamos que $\{ {\it I} \land B \land V_0 = f_v\}$ S $\{f_v < V_0\}$ es una tripla de Hoare válida

\vspace{0.3cm}

Definimos $f_v$ como $f_v = |eventos| - i$
\begin{equation}
    \begin{aligned}
        & |eventos| - i < V_0 \\
        & wp(i:= i + 1, |eventos| - i < V_0) = def(i) \land |eventos| -(i+1) < V_0 \\
        & wp(i:= i + 1, |eventos| - i < V_0) = 0 \leq i \leq |eventos| \land |eventos| - i - 1 < V_0 \\
        & wp(i:= i + 1, |eventos| - i < V_0) = 0 \leq i \leq |eventos| \land |eventos| - i \leq V_0 \\
        & wp(i:= i + 1, |eventos| - i < V_0) = 0 \leq i \leq |eventos| \land f_v \leq V_0
    \end{aligned}
\end{equation}

Luego, verificamos que ${\it I} \land B \land V_0 = f_v \implies 0 \leq i \leq |eventos| \land f_v \leq V_0$

\begin{equation}
    \begin{split}
        &\resizebox{.9\hsize}{!}{$0 \leq i \leq |eventos| \yLuego res = recurso(apuesta_c * pago_c)^{\#apariciones(subseq(eventos,0,i),True)}*(apuesta_s * pago_s)^{\#apariciones(subseq(eventos,0,i),False)}$} \\
        & \land i < |eventos| \land V_0 = f_v \implies 0 \leq i \leq |eventos| \land f_v \leq V_0
    \end{split}
\end{equation}

\begin{equation}
    \begin{aligned}
        &0 \leq i < |eventos| \land V_0 = f_v \implies 0 \leq i \leq |eventos| \land f_v \leq V_0 \\
        &\therefore {\it I} \land B \land V_0 = f_v \implies 0 \leq i \leq |eventos| \land f_v \leq V_0
    \end{aligned}
\end{equation}

\vspace{0.3cm}

\subsection{Axioma 5}

Axioma 5: Verificamos que ${\it I} \land f_v \leq 0 \implies \neg B$

\begin{equation}
    \begin{split}
        &\resizebox{.9\hsize}{!}{$0 \leq i \leq |eventos| \yLuego res = recurso(apuesta_c * pago_c)^{\#apariciones(subseq(eventos,0,i),True)}*(apuesta_s * pago_s)^{\#apariciones(subseq(eventos,0,i),False)}$} \\
        &\land |eventos| - i \leq 0 \implies i \geq |eventos|
    \end{split}
\end{equation}

\begin{equation}
    \begin{split}
        &0 \leq i \leq |eventos| \land |eventos| \leq i \implies i \geq |eventos| \\
        &i = |eventos| \implies i \geq |eventos|
    \end{split}
\end{equation}
que es lo que se quería probar. De esta forma sabemos que el ciclo y por ende el programa, termina.

\newpage

Habiendo demostrado entonces, que a partir de un estado $P_c$ el programa es correcto y termina, es necesario ahora demostrar que siempre que se ejecute el programa y se cumpla el requiere,
se llegará a un estado que implique $P_c$. \\
Es decir, demostrar que vale la tripla $\{P\}$ S $\{P_c\}$, siendo P el requiere de la especificación y S el resto del programa.

\vspace{0.3cm}

$P:\{apuesta_c + apuesta_s = 1 \land pago_c > 0 \land pago_s > 0 \land apuesta_c > 0 \land apuesta_s > 0 \land recurso > 0\}$

\vspace{0.3cm}

$S:res := recurso$ ; $i := 0$

\vspace{0.3cm}

$P_c:\{recurso > 0 \land res = recurso \land i = 0\}$

\vspace{0.3cm}

$P \implies wp(S,P_c)$\\

\vspace{0.3cm}

$wp(res := recurso; i:=0, (recurso > 0 \land res = recurso \land i = 0))$

\vspace{0.3cm}
$wp(res := recurso, wp(i:=0, (recurso > 0 \land res = recurso \land i = 0))$

\vspace{0.3cm}
$wp(res := recurso, (recurso > 0 \land res = recurso \land 0 = 0))$

\vspace{0.3cm}
$recurso > 0 \land recurso = recurso \land 0 = 0$

\vspace{0.3cm}
$(apuesta_c + apuesta_s = 1 \land pago_c > 0 \land pago_s > 0 \land apuesta_c > 0 \land apuesta_s > 0 \land recurso > 0) \implies recurso > 0$

\vspace{0.3cm}
Queda entonces demostrado que $P \implies P_c$ y por consiguiente, que el programa es correcto.

\end{document}
